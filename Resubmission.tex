\documentclass{llncs}
\usepackage{amsmath}
\usepackage[show]{ed}
\usepackage{xspace}
\usepackage{paralist} % Used for the compactitem environment which makes bullet points with less space between them
%\usepackage{etoolbox}
\usepackage{eurosym}
\usepackage{graphicx}
\usepackage{wrapfig}
\def\stex{\texorpdfstring{\raisebox{-.5ex}S\kern-.5ex\TeX}{sTeX}\xspace}
\def\sTeX{\stex}
\usepackage{hyperref}

\def\degree{\ensuremath{^\circ}}
\newcommand{\sys}{\textsc{RPresentation}\xspace}

%----------------------------------------------------------------------------------------
%	TITLE SECTION
%----------------------------------------------------------------------------------------

\title{Relational Presentations Using Semantic Closeness\\ Spatial Narrative for Mathematical Content \thanks{{\it Copyright \copyright\, 2015 by the paper's authors. Copying permitted only for private and academic purposes.} In: R. Bergmann, S. G{\"o}rg, G. M{\"u}ller (Eds.): Proceedings of the LWA 2015 Workshops: KDML, FGWM, IR, and FGDB. Trier, Germany, 7.-9. October 2015,  published at http://ceur-ws.org}}
\renewcommand\footnotemark{}

\author{Naomi Pentrel, Michael Kohlhase}
\institute{Jacobs University Bremen}
%----------------------------------------------------------------------------------------
% Macros
%----------------------------------------------------------------------------------------

%----------------------------------------------------------------------------------------

\begin{document}
\maketitle

%----------------------------------------------------------------------------------------
%	ARTICLE CONTENTS
%----------------------------------------------------------------------------------------

\begin{abstract}
Visualization of knowledge is important to foster learning. Especially so in Mathematics where students have to understand not just one topic at a time but also the related concepts. Taking the typical Mathematics lecture as an example, it is often the case that students come from different backgrounds. When a new topic is introduced it would therefore be ideal to have a simple way to find dependencies and present students with an easy way to catch up on topics they have not learned. To optimize the visualization of information and its interdependencies, a way to present information without losing context is therefore necessary. The following research takes an existing annotated corpus and presents its contents while allowing students to see dependencies between topics and encouraging them to explore related mathematical concepts. Thus students can interactively learn the concepts the current topic depends on by taking small detours through those topics, should they need to refresh their memories. This approach to presenting learning materials changes how we interact with course materials and it is ultimately applicable to almost all areas in which knowledge needs to be transferred.
\end{abstract}

Original Publication at:\\ \href{http://www.cermat.org/events/MathUI/15/proceedings/Pentrel-Kohlhase_Semantic_Closness-MathUI_15.pdf}{http://www.cermat.org/events/MathUI/15/proceedings/Pentrel-Kohlhase\\\_Semantic\_Closness-MathUI\_15.pdf}

\end{document}

%%% Local Variables:
%%% mode: latex
%%% TeX-master: t
%%% End:
