\documentclass[twoside, 12pt]{article}
\usepackage{amsmath}
\usepackage{lipsum} % Package to generate dummy text throughout this template
\usepackage[english]{isodate}% http://ctan.org/pkg/isodate
\usepackage{pgfgantt}
\usepackage[show]{ed}
\setlength{\parindent}{0pt}
\usepackage[sc]{mathpazo} % Use the Palatino font
\usepackage[T1]{fontenc} % Use 8-bit encoding that has 256 glyphs
\linespread{1.2} % Line spacing - Palatino needs more space between lines
\usepackage{microtype} % Slightly tweak font spacing for aesthetics
\usepackage{minted}
\usepackage[section]{placeins}
\usepackage{nameref}
\usemintedstyle{autumn}
\usepackage[hmarginratio=1:1,top=32mm,left=25mm,right=25mm,columnsep=18pt,bottom=27mm]{geometry} % Document margins
\usepackage{multicol} % Used for the two-column layout of the document
\usepackage[hang, small,labelfont=bf,up,textfont=it,up]{caption} % Custom captions under/above floats in tables or figures
\usepackage{booktabs} % Horizontal rules in tables
\usepackage{float} % Required for tables and figures in the multi-column environment - they need to be placed in specific locations with the [H] (e.g. \begin{table}[H])
\usepackage{hyperref} % For hyperlinks in the PDF
\usepackage{xspace}
\usepackage{lettrine} % The lettrine is the first enlarged letter at the beginning of the text
\usepackage{paralist} % Used for the compactitem environment which makes bullet points with less space between them
\usepackage{etoolbox}
\usepackage{eurosym}
\patchcmd{\thebibliography}{\section*{\refname}}{}{}{}
\usepackage{abstract} % Allows abstract customization
\renewcommand{\abstractnamefont}{\normalfont\bfseries} % Set the "Abstract" text to bold
\renewcommand{\abstracttextfont}{\normalfont\small\itshape} % Set the abstract itself to small italic text

\usepackage{titlesec} % Allows customization of titles
%\renewcommand\thesection{\Roman{section}} % Roman numerals for the sections
%\renewcommand\thesubsection{\Roman{subsection}} % Roman numerals for subsections
\titleformat{\section}[block]{\large\scshape\centering\bfseries}{\thesection.}{1em}{} % Change the look of the section titles
\titleformat{\subsection}[block]{\large\scshape}{\thesubsection.}{1em}{} % Change the look of the section titles
\usepackage{cite}
\usepackage{fancyhdr} % Headers and footers
\pagestyle{fancy} % All pages have headers and footers
\fancyhead{} % Blank out the default header
\fancyfoot{} % Blank out the default footer
\fancyhead[C]{Naomi Pentrel $\bullet$ \href{mailto:n.pentrel@jacobs-university.de}{n.pentrel@jacobs-university.de} $\bullet$ \today} % Custom header text
\fancyfoot[RO,LE]{\thepage} % Custom footer text
\usepackage{graphicx}
\usepackage{wrapfig}

\usepackage{tocloft}

%\usepackage{csquotes}
%\MakeOuterQuote{"}

\hypersetup{
    bookmarks=true,
    unicode=false,
    pdftoolbar=true,
    pdfmenubar=true,
    pdffitwindow=false,
    pdfstartview={FitH},
    pdftitle={My title},
    pdfauthor={Author},
    pdfsubject={Subject},
    pdfcreator={Creator},
    pdfproducer={Producer},
    pdfkeywords={keyword1} {key2} {key3},
    pdfnewwindow=true,
    colorlinks=false,
    %linkcolor=red,
    %citecolor=green,
    %filecolor=magenta,
    %urlcolor=cyan
    pdfborder={0 0 0},
}


\usepackage{caption}
\usepackage{subcaption}
\usepackage[section]{placeins}
\usepackage{qtree}
\usepackage{gensymb}

\usepackage{setspace}

\usepackage{color}

%\doublespacing
% or:
%\onehalfspacing

%\renewcommand\cftchapafterpnum{\vskip10pt}
%\renewcommand\cftsecafterpnum{\vskip15pt}

%----------------------------------------------------------------------------------------
%	TITLE SECTION
%----------------------------------------------------------------------------------------

\title{\vspace{-15mm}\fontsize{24pt}{10pt}\selectfont\textbf{// All comments are NOT created equal}} % Article title


\newenvironment{myfont}{\fontfamily{\sfdefault}\selectfont}{\par}

%----------------------------------------------------------------------------------------
% Macros
%----------------------------------------------------------------------------------------
\newcommand{\sys}{\textsc{Knowtation}\xspace}

%----------------------------------------------------------------------------------------

\begin{document}
\thispagestyle{empty}
\pagenumbering{roman}
\begin{flushright}
    \includegraphics[scale=1.0]{assets/Logo}
  \end{flushright}
  \vspace{20mm}
  \begin{center}
    \huge
    \textbf{The Combination of Spatial Narrative and Semantic Closeness to derive Visualizations of Theory Graphs}
  \end{center}
  \vspace*{4mm}
  \begin{center}
   \Large by
  \end{center}
  \vspace*{4mm}
  \begin{center}
    \Large
    \textbf{Naomi Pentrel}
  \end{center}
  \vspace*{20mm}
  \begin{center}
    \large
    Bachelor Thesis Proposal in Computer Science
  \end{center}
  \vfill
  \begin{flushright}
    \large
    \begin{tabular}{l}
      
      \hline
      Prof. Dr. Michael Kohlhase \\
      \\
    \end{tabular}
  \end{flushright}
  \vspace*{8mm}
  \begin{flushleft}
    \large
    Date of Submission: \today \\
    \rule{\textwidth}{1pt}
  \end{flushleft}
  \begin{center}
    \Large Jacobs University Bremen - School of Engineering and Science
  \end{center}

\newpage
\noindent
  With my signature, I certify that this thesis proposal has been written by me
  using only the indicated resources and materials. Where I have
  presented data and results, the data and results are complete,
  genuine, and have been obtained by me unless otherwise acknowledged;
  where my results derive from computer programs, these computer
  programs have been written by me unless otherwise acknowledged. I
  further confirm that this thesis has not been submitted, either in
  part or as a whole, for any other academic degree at this or another
  institution.

  \vspace{20mm}

    \includegraphics[scale=0.2]{assets/Signature}
 \hfill Bremen, \today
  
\newpage

\thispagestyle{fancy} % All pages have headers and footers

%----------------------------------------------------------------------------------------
%	ARTICLE CONTENTS
%----------------------------------------------------------------------------------------

 \section*{Abstract}
 \label{sec:abstract}
Visualization of knowledge is important to foster learning. To optimize the visualization of information and its interdependencies, we will focus on identifying ways to present information without loosing context. The proposed research aims at taking an existing annotated corpus and presenting its contents in an interactive way that allows the audience to see the dependencies of the covered topics. This form of presenting knowledge should enable users to interactively examine materials that are related to the topics the audience is interested in.\\

Taking the typical lectures as an example, it is often the case that, especially towards the end of the semester, students have difficulties remembering earlier topics. When a new topic is introduced it would be ideal to have a simple way to find dependencies and present students with an easy way to catch up.\\

The OMDoc (Open Mathematical Documents) format \cite{Kohlhase:OMDoc1.2} is a content markup scheme for mathematical documents. Using OMDoc, the proposed research aims at visualizing the connections of different pieces of mathematical concepts in an interactive way that allows students to learn the concepts the current topic depends on, should they need to refresh their memories. It will examine if and how spatial narrative and semantic closeness can be combined to foster learning. If this approach proves successful it will transform how we interact with course material and learning processes in general. Ultimately, this research may be applicable to almost all areas in which knowledge needs to be transferred.\\ 

  \newpage
  \tableofcontents

  \clearpage
  \pagenumbering{arabic}

  \section{Introduction}
  \label{sec:introduction}

The intention behind the Semantic Web \cite{BernersLee:tsw98} is to transform the World Wide Web (WWW) from a web of mere human readable information to an information web with relations between different resources. There are interesting similar developments in other fields such as that of new presentation tools. These aim to enable the presentation of information in a more interconnected way that provides context. Both developments strive to connect pieces of information through providing context, to not just provide information, but rather knowledge.\\

The concept behind these two developments is quite similar. Looking at the history of the WWW, specifically at the era of Web 2.0 \cite{Weller:npentrel14}, we notice that the web changed from an entity that only few people could add information to, to an entity that enables everyone to generate content. While this led to a lot of information on the web, it provided comparatively little knowledge. Here the processing of information to provide knowledge is lacking. Similarly the existence of tools like Powerpoint have lead to virtually everyone creating presentations in this format. As everyone who has witnessed many of these presentations knows, these are often easily compared to blogs: they contain some pieces of information but often lack structure and context.\\

In essence, both issues stem from information being generated without enough context or structure. This makes the information hard to process by humans and/or machines. The Semantic Web hopes to change the former, whereas new, more visual presentation tools, using spatial narrative, provide a framework for the creation of presentations that are easier to process for humans and thus facilitate knowledge transfer. The work proposed here intends to provide a tool for automatic contextual visualization of information through the usage of spatial narrative and semantic closeness of information. It adapts some of the concepts of the semantic web on a smaller scale.\\

\begin{figure}[H]
        \centering
                \fbox{\includegraphics[width=0.8\textwidth]{assets/model2}}
                \caption{From Data to Knowledge}
                \label{fig:dataknow}
\end{figure}

One can define knowledge as "the information necessary to support intelligent reasoning." \cite{Kohlhase:Complog:base} That said, it is imperative to facilitate knowledge transfer. Figure \ref{fig:dataknow} (based on \cite{ProbstRaubRomhardt} and \cite{Kohlhase:Complog:base}) shows the path from mere data to knowledge, through adding context, such that we gain information. From information, we can obtain knowledge through adding networking.\\

We want to go from merely displaying information to visualizing information by displaying the network/context a piece of information is part of. In this we will not try to semantically represent the meaning of the information that is shown but rather to show the network, i.e. the context the information is a part of. Specifically we want to examine whether spatial narrative can be combined with inherent semantic closeness of different pieces of information to facilitate knowledge transfer.\\

To achieve this goal, we intend to create a knowledge presentation system \sys to visually present information and its context (and networks). For this we will be using the OMDoc (Open Mathematical Documents) format which will provide us with an annotated set of documents. This will give us information about the semantic closeness of different pieces of information. Taking this as a starting point, we will visualize the dependencies of the information contained in these documents. Of course we will not visualize the dependencies as a simple graph but use the approach of spatial narrative with the help of a presentation framework such as Impress.js in order to show the current piece of information of interest within the network of dependent information.\\

\begin{figure}[H]
        \centering
        \begin{subfigure}[b]{0.47\textwidth}
                \fbox{\includegraphics[width=\textwidth]{assets/Annie1}}
			   \caption{Simple, explanatory slide}
                \label{fig:Annie-presentation-teacher}
        \end{subfigure}%
        \qquad %add desired spacing between images, e. g. ~, \quad, \qquad, \hfill etc.
          %(or a blank line to force the subfigure onto a new line)
        \begin{subfigure}[b]{0.47\textwidth}
                \fbox{\includegraphics[width=\textwidth]{assets/Annie2}}
		  \caption{Information dependencies}
                \label{fig:Annie-presentation-information}
        \end{subfigure}
        \caption{Pythagorean Theorem}\label{fig:PythagoreanTheorem}
\end{figure}

To exemplify this: Annie, a young student learning mathematics, is watching her teacher give a presentation on how to use the Pythagorean theorem (see figure \ref{fig:Annie-presentation-teacher}). She is given a triangle with sides a = 3 cm and b = 4 cm. Now she wants to use the theorem to calculate the length of side c. Annie already knows how to calculate the square of a number and thus calculates that $c^2 = 25 \implies c = 5$. However Annie made a mistake and did not know what a right triangle is. Her teacher tells her that the question was a trick question and that the answer is wrong because the triangle she calculated this for is not a right triangle. Now Annie has to try to find out what a right triangle is. This shows the dependencies of different pieces of information (see figure \ref{fig:Annie-presentation-information}), i.e. the Pythagorean theorem depends on knowledge of squares,triangles, lengths, area calculation, and right angles. In an ideal presentation this information should have been linked to the current piece of information so that Annie could have directly looked it up instead of having to search for it.\\

After this contrived story, it is left to the reader to think about examples in his or her own life where he or she would have appreciated having the context of the information he or she was learning at the time. There are many stories such as Annie's in real life and it is therefore important to provide this context to facilitate learning and knowledge transfer as a whole. \\

\section{Preliminaries}
\label{sec:preliminaries}

To understand the problem of information that lacks context, this part of the proposal will provide the necessary background information to allow us to understand why combining spatial narrative with inherent semantic closeness might be a solution to allow Annie's teacher to transfer knowledge more effectively. While doing so we will demonstrate why this research is important and how it will improve the transfer of knowledge.\\

\subsection{Terminology}
\label{sec:terminology}

In this section we will introduce some basic terminology which will be needed to discuss the actual subject matter in the following part. We will conceptualize what we mean by the terms information, knowledge, and context.\\

The Oxford Dictionary of English \cite{OED:npentrel14} defines \textbf{information} as "what is conveyed or represented by a particular arrangement or sequence of things". Recalling figure \ref{fig:PythagoreanTheorem} and our contrived story of Annie, an example for information is the sequence \textit{$a^2 + b^2 = c^2$}. For the term \textbf{Knowledge}, we will use the definition in Merriam-Webster's thesaurus \cite{Webster:npentrel14} which defines it as the "fact or condition of knowing something with familiarity gained through experience or association". This \textbf{association} is of special importance for the current project. We will define it as knowledge that is made up out of several pieces of information that are connected. In the example from above knowledge would be knowing when and how the Pythagorean theorem can be used by knowing the information that the usage of the Pythagorean theorem depends on (see figure \ref{fig:Annie-presentation-information}).\\ 

\begin{wrapfigure}{r}{0.5\textwidth}
\vspace{-26pt}
  \begin{center}
  \fbox{
    \includegraphics[width=0.45\textwidth]{assets/Annie3}
    }
  \end{center}
\vspace{-20pt}
  \caption{Information in context}
  \label{fig:Annie-context}
\vspace{-10pt}
\end{wrapfigure}

According to the Merriam-Webster thesaurus \cite{Webster:npentrel14}, \textbf{context} is defined as "the parts of a discourse that surround a word or passage and can throw light on its meaning". Context can be easily explained with the example of Annie learning the Pythagorean theorem. The information about the calculation of the length of the third side of the triangle is only helpful for Annie within the context of the pieces of information this depends on. Without this context, Annie cannot understand the theorem. Figure \ref{fig:Annie-context} shows the context by showing the dependencies of the theorem, i.e. calculating squares, area calculation, triangles, and right angles. It semiotically represents relatedness of concepts by linking the different concepts with lines, thus creating associations between the different pieces of information. We will be using the terms context and network almost interchangeably and depend on the reader to have the right intuition about these terms.\\

\subsection{OMDoc}
\label{sec:OMDoc}

OMDoc (\textbf{O}pen \textbf{M}athematical \textbf{Doc}uments) \cite{Kohlhase:OMDoc1.2} is an XML-based system that provides a data model and a format for content markup for mathematical documents. As such it is a semi-formal domain ontology. An ontology provides the framework for creating a semantic structure. It formally describes concepts within a specified area. A semi-formal ontology \cite{Sheth:npentrel14} is an ontology where formality of semantics is not a given. The semi-formal ontology can consist of partial or incomplete knowledge. \\

If OMDoc was used in every part of mathematics (and related fields), we would have a repository of mathematical knowledge that could be processed in different ways. This is possible because OMDoc provides the framework to create and store mathematical objects such as definitions and concepts. The relations between different mathematical objects and the attributes of mathematical objects can be added through annotations. This elevates the net of separate pieces of information that are stored individually to a semantic representation and one step further to knowledge.\\

\begin{wrapfigure}{l}{0.6\textwidth}
\vspace{-26pt}
  \begin{center}
  \fbox{
    \includegraphics[width=0.55\textwidth]{assets/Annie4xml3}
    }
  \end{center}
\vspace{-20pt}
  \caption{Narrative Layers and semantic closeness}
  \label{fig:Annie-semantic-closeness}
\vspace{-10pt}
\end{wrapfigure}

In \textit{A mathematical approach to ontology authoring and
documentation} \cite{LK:MathOntoAuthDoc09}, it is stated that "documents consist of narrative and content layers". In our case, the content layers are the mathematical objects, i.e. the statements or theories. Narrative layers refer to the order the mathematical objects from content layers are presented. In figure \ref{fig:Annie-semantic-closeness}, this temporal information is represented through the arrows that connect the topics as the lecturer intends to teach them. Through the annotations within the document, we can establish levels of semantic closeness for different mathematical objects. These levels of semantic closeness are represented through the different sizes of the frames of the related information. In the proposed research the mathematical knowledge will be processed and visualized in a way that enables students to learn more efficiently through combining spatial narrative and semantic closeness. Thus it will not only provide the information or provide a narrative but it will provide a more interactive approach to presentations and learning.\\

\subsection{Status of Information within the (Dis)Course}
\label{sec:infostatus}

In linguistics the concept of information packaging \cite{CambridgeGrammar:npentrel14} is well known and widely discussed. Within the study of \textit{information structure}, one discriminates between \textit{familiar/ old} and \textit{unfamiliar/ new} information. \textit{Familiar/ old} information is shared by speaker and addressee, i.e. it is in the intersection of knowledge of student and teacher. \textit{New/ unfamiliar} information is not in the shared knowledge base or the content commons \cite{CNX:whitepaper}. In addition one distinguishes information that is old or new with respect to the discourse or with respect to the addressee.

\begin{center}
\textit{"My sister went to the circus the other day; \underline{she} said \underline{it} was brilliant."}\\
\end{center}

In this example in the first part of the sentence, \textit{discourse-new} information pertaining to my sister and to a circus are introduced. In the second part, the underlined parts are considered \textit{discourse-old} since they have already been introduced. These terms are coined to refer to the accessibility of the information to participants of the discourse \cite{Newness:npentrel14}. The accessibility depends on the relative \textit{newness}, i.e. recency of mention, of this information.\\

These concepts can be adapted to the situation of teaching mathematics or computer science to students in a class. In general we will call the information that the speaker is sharing with the addressees/students \textit{course-new}. Since we are in the setting of a university course, the \textit{course-new} information will generally depend on information that is \textit{course-old}. We will additionally introduce a third modus for information called \textit{course-ancient}. This covers the situation where the addressee has difficulties following the speaker since the \textit{course-new} information depends on \textit{course-old} information that might be 'too old' to be easily remembered, i.e. \textit{course-ancient}.\\

\begin{wrapfigure}{r}{0.5\textwidth}
\vspace{-26pt}
  \begin{center}
  \fbox{
    \includegraphics[width=0.45\textwidth]{assets/Annie4xml3ancient}
    }
  \end{center}
\vspace{-20pt}
  \caption{Course-ancient information}
  \label{fig:Annie-ancient}
\vspace{-10pt}
\end{wrapfigure}

Going back to the example of Annie, the information about the Pythagorean theorem that the teacher is just introducing would be considered \textit{course-new}. The information about the right angle which Annie cannot access anymore since she was taught about this too long ago is \textit{course-ancient}. This is marked orange in \ref{fig:Annie-ancient}. The information about squares which Annie still remembers is \textit{course-old}. Through the semantic closeness that OMDoc provides us we will be able to determine the status of information within the (dis)course. \\

\subsection{Different Representations of Information}
\label{sec:inforep}

Information can be (re)presented in different ways: text, visuals, images, etc. Actually everything that is in the world around us conveys information to us. A sign of a women on a door conveys to most of us that this is a bathroom for women. The information graphs that OMDoc provides us with could also be simply visualized as a tree of information. But this would not be very engaging and it would be hard for a human to process. Similarly traditional slide-based presentations do not provide a very engaging or interactive form of presenting information. \\  

In the proposed research we will focus less on text, tree-like structures, metaphors, or semiotics like the sign of a woman on a door. We will rather focus on the spatial narrative and placing content that inherently belongs together visually close together since this can convey the meaning that these objects belong together.\\

\begin{figure}[H]
        \centering
        \begin{subfigure}[b]{0.3\textwidth}
                \fbox{\includegraphics[width=\textwidth]{assets/size-of-objects}}
                \caption{Importance of Objects}
                \label{fig:imp_objects}
        \end{subfigure}%
        \qquad %add desired spacing between images, e. g. ~, \quad, \qquad, \hfill etc.
          %(or a blank line to force the subfigure onto a new line)
        \begin{subfigure}[b]{0.3\textwidth}
                \fbox{\includegraphics[width=\textwidth]{assets/relation}}
                \caption{Relation of Objects}
                \label{fig:rel_objects}
        \end{subfigure}
        \caption{Conveying Visual Meaning}\label{fig:Objects}
\end{figure}

The figure \ref{fig:Objects} above clarifies what we mean by 'meaningful representations'. Figure \ref{fig:imp_objects} shows how size conveys importance. It is well known that people tend to identify the circle on the right as most important, the ones on the left are less important due to their relative size. Considering the relation of the three circles most people say that the two circles on the left belong together whereas the circle on the right is clearly something different. When shown Figure \ref{fig:rel_objects} they quickly change their mind and say that they are now actually related.\\

These perceptions can be used to convey meaning visually. For the example of Annie this means that the \textit{course-new} information would likely be important and therefore bigger than the \textit{content-old} or \textit{content-ancient} information. From there on it depends how important the topics related to this topic are and we will use different sizes to differentiate between more important and less important topics. Similarly we will employ visual relations as in \ref{fig:rel_objects} to show related content. Thus we will provide Annie with a meaningful representation of the information taken from the information graph. In turn this should allow Annie to absorb information about importance and relations instinctively which improves knowledge transfer.\\

\subsection{Primitives}
\label{sec:primitives}

Within knowledge representation, five evaluation criteria for knowledge representation are known \cite{Kohlhase:Complog:base}: Expressive Adequacy, Reasoning Efficiency, Primitives, Meta-representation, and Incompleteness. Primitives are the different elements of representation. In terms of evaluation it is important to have intuitive primitive elements. \\

For Impress.js, the primitive elements are text, images, slide-like structures (frames), and the visual relation between the different frame-like structures. The exact layout of the frames and the best visualization of the relation between the frames is to be determined. Ideally this should lead to Annie having the information about the Pythagorean theorem on a slide-like structure while having all the information the Pythagorean theorem depends on easily accessible so that she can reach \textit{course-old} information easily.  \ednote {not sure whether I understand this correctly - does this need more explanation?}
\ednote {visual and graph language have different primitives and it's interesting how it is mapped to one another}

\subsection{Spatial Narrative}
\label{sec:spatialnarrative}

Prezi is "a virtual whiteboard that transforms presentations from monologues into conversations: enabling people to see, understand, and remember ideas" \cite{Prezi:npentrel14}. When doing a good prezi presentation one has to understand the topic one is presenting on a deeper level and think about how one can portray connections between content visually. The development of presentation tools like Impress.js and Prezi, has changed how people think about presentations while making use of spatial narrative.\\

The concept of Spatial Narrative stems from the establishment of frameworks for "the creation of computer-assisted flexibile 'guided tours' based on the thematically and narrative linking of a set of locations within an area into a 'spatial narrative'" \cite{SpatialNarratives:npentrel14}. Teachers that are presenting a topic are an example for the experience of a tour guide for a certain topic.\\

The benefits of spatial narrative is that the audience gains a more thorough understanding of the subject and has a better understanding of the interconnections between different pieces of information. By providing the audience with a story as a narrative one also makes use of the concept of Storytelling which is largely accredited with the benefit that the audience will have an easier time following the presenter. That is because our brains are not made to memorize a lot of unconnected pieces of information; it is far easier for us to remember information if it comes in story form \cite{Storytelling:npentrel14}.\\

In this research we will use this knowledge to form an interconnected web of information that tells a visual story based on semantic closeness to facilitate the transfer of knowledge. For Annie this means that she would be presented with the information about the Pythagorean theorem in a way that the \textit{course-ancient} and \textit{course-old} information this \textit{course-new} information depends on would be visually close and visually connected. It also means that Annie's teacher can answer Annie's questions easily by just going to the related \textit{course-ancient} information in a click.\\

\subsection{Related Work}
\label{sec:relatedworks}

When searching for related works there is little to be found. Some works have focused on making teaching more effective by providing students with all lecture materials directly after class which has been found to have a positive influence on learning \cite{DBLP:dblp_journals/tochi/BrothertonA04}. For this systems that can automatically capture and index classroom events as videos and images have been devised \cite{indexedclass:npentrel14}. Some of these systems use temporal information \cite{DBLP:dblp_journals/isci/ChungS97} to show multiple resources in sync as they appeared in the lecture. However, while this approach also aims at improving the learning experience, it does not in any way link content semantically based on dependencies or allow for more interactive presentations through combining semantic closeness and spatial narrative.\\

Another piece of research has focused on the automatic generation of mind maps from text \cite{abdeen2009direct}. The software they developed uses Semantic Web technologies to create mind maps that show the relations between nodes that represent different objects. While the proposed research also intends to show the relation between pieces of information, it does it on a different level. For instance, it can represent that Shakespeare lived in Stratford but it would not be able to show the dependencies of mathematical concepts as it is limited to representing nodes in single words within the mind map. We, on the other hand, will focus on exactly that, i.e. dependencies and the creation of interactive presentations.\\


\section{Work Plan}
\label{sec:workplan}

\subsection{WP1: Understanding the API of Impress.js}
\label{sec:wp1}

The first step to implementing \sys is to understand the tool we want to represent the knowledge in later. Since Prezi at the current time does not have an Application Programming Interface (API) that would allow for the generation of a whole presentation without using the editor, this research will employ another tool that is inspired by the idea behind Prezi. Bartek Szopka's \textbf{Impress.js} \cite{JSImpress:npentrel14} is an open-source presentation framework which is based on CSS3 transforms and transitions.\\

Impress.js gives us all the necessary primitives that were outlined in section \ref{sec:primitives}. Therefore it will give us the perfect environment to transform the XML data that OMDoc provides us with into an interactive presentation. In this step we will define several models of how data is presented within a frame. Next we will automate that the different frames have different sizes according to the importance of the presented information (see section \ref{sec:inforep}). Last but not least we need to create several models of how the relations between the different frames can be portrayed within Impress.js.\\

After finishing this part we should have several different presentation styles from which one can be chosen to continue the main research question.\\ 

\ednote{look at Impressionist and jimpress when writing actual thesis}\\

\subsection{WP2: Creation of Ordered Information Graphs using OMDoc}
\label{sec:wp2}

The idea is to use the General Computer Science Lecture notes \cite{Kohlhase:GenCSI:base} as an experiment. The General Computer Science Lecture notes are an annotated set of documents. Using OMDoc, we will create an XML information graph out of the annotated set of documents.\\

The information graph we receive from OMDoc should be topologically sorted based on dependencies of knowledge. This will allow us to assign values representing the status of information within the course as as in section \ref{sec:infostatus}. It will also allow us to create values for the relative semantic closeness of information which will be needed for creating a spatial narrative as outlined in section \ref{sec:spatialnarrative}. Knowing the dependencies and the status of the information will allow us to choose which information to show the user in the close surroundings of the current information we are showing.\\

Each piece of information needs to also have an importance level assigned to it as well (see section \ref{sec:inforep}). This ought to be determined by knowing how often the concept is used and how important it is for the student to grasp it.\\

\subsection{WP3: Combination OMDoc and Spatial Narration}
\label{sec:wp3}

Once we have an ordered information graph with all the additional attribute values mentioned in section \ref{sec:wp2}, we will create the knowledge presentation system \sys. \sys will visualize the information from the graph. The outcome of this part should be a general \sys presentation model, that implements the models according to the models chosen in section \ref{sec:wp1}. We will then use \sys to visualize the information in WP4.\\

\subsection{WP4: Implementation of OMDoc Examples in Impress.js}
\label{sec:wp4}

\sys will take the ordered information graph of the General Computer Science Lecture notes \cite{Kohlhase:GenCSI:base} with all the additional attributes we generated in \ref{sec:wp2} and transform it into a spatial presentation using Impress.js. This will be done by parsing the created XML and creating the markup that Impress.js is using to create the chosen presentation models from section \ref{sec:wp1}. After this work package, we should have one (or more) presentations of the topics covered in the General Computer Science Lecture notes.\\

It is a stretch goal for \sys to create or change these presentations on the fly as needed by the audience. The specific use case would be when Annie asks her teacher about right angles. The teacher would then just have to click on right angles and \sys would create a short guided tour through the topic of right angles before returning to the subject of the Pythagorean theorem.\\

\subsection{WP5: Qualitative Analysis and Evaluation of Findings}
\label{sec:wp5}

Once we have examples, we will undertake a qualitative analysis of the visualization produced by the implemented knowledge presentation system \sys . To achieve this, we will use the newly created examples from \ref{sec:wp4} and the general General Computer Science Lecture notes \cite{Kohlhase:GenCSI:base}. We will have two groups: a control group that is shown the lecture notes and a test group that is shown the new, interactive Impress.js presentations. Afterwards both groups have to answer a set of questions. In addition, we will determine whether the implemented \sys meets the goals outlined in the Evaluation Criteria (see section \ref{sec:evalcriteria}).\\

\section{Evaluation Criteria}
\label{sec:evalcriteria}

The evaluation criteria for the research at hand are on the one hand systematic and on the other hand cognitive. Systematically, we will be evaluating two main challenges: First we will evaluate whether the knowledge presentation system \sys that was developed allows for the creation of a full presentation from a given annotated document. Then we will evaluate the speed of \sys. Special attention will be given to the question whether this could be used in real time such that a student could request a presentation on a certain topic and whether this could be generated on the fly within seconds.\\

Cognitively, we will be exploring the space of information systems. Therefore we will be examining whether the developed system provides a better system for knowledge transfer. For this a qualitative analysis will be conducted as outlined in section \ref{sec:wp5}. Based on that we will be able to understand the positive and possibly negative effects of the combination of Spatial Narrative and OMDoc within \sys .\\

Furthermore, with the above evaluation, we want to answer the main question of this proposed research, namely how and whether spatial narrative can be combined with inherent semantic closeness to facilitate and improve knowledge transfer.

If \sys can create a clearly structured presentation from a given annotated document that provides a better knowledge transfer experience, this research will be a full success. If this should fail and we know why it did not or cannot work the research will have succeeded in exploring an interesting way of trying to improve knowledge transfer and will serve to show future researchers which other parameters need to be kept in mind in order to succeed in future.

A stretch goal of this project is to have this research accessible for common users that could use it to create their own presentations. Another stretch goal is for \sys to create or change presentations on the fly as outlined in \ref{sec:wp4}. It is not the goal of this research that users should be able to reconstruct the original graph from the presentation.\\

\section{Timeline}
\label{sec:timeline}

\begin{figure}[H]
\begin{center}
\begin{ganttchart}[y unit title=0.4cm,
y unit chart=0.5cm,
vgrid,hgrid, 
title label anchor/.style={below=-1.6ex},
title left shift=.05,
title right shift=-.05,
title height=1,
bar/.style={fill=gray!50},
incomplete/.style={fill=white},
progress label text={},
bar height=0.7,
group right shift=0,
group top shift=.6,
group height=.3,
group peaks={}{}{.2}]{24}
%labels
\gantttitle{Proposed Timeline}{24} \\
\gantttitle{December}{4} 
\gantttitle{January}{4} 
\gantttitle{February}{4} 
\gantttitle{March}{4} 
\gantttitle{April}{4} 
\gantttitle{May}{4} \\
%tasks
\ganttbar{\ref{sec:wp1} WP1}{2}{8} \\
\ganttbar{\ref{sec:wp2} WP2}{9}{12} \\
\ganttbar{\ref{sec:wp3} WP3}{11}{14} \\
\ganttbar{\ref{sec:wp4} WP4}{15}{18}\\
\ganttbar{\ref{sec:wp5} WP5}{19}{21.5}
%relations 
\ganttlink{elem0}{elem1} 
\ganttlink{elem1}{elem2} 
\ganttlink{elem2}{elem3} 
\ganttlink{elem3}{elem4} 
\end{ganttchart}
\end{center}
\caption{Gantt Chart}
\end{figure}

\textbf{Milestones}
\begin{itemize}
\item \textbf{Until February 1st}: \nameref{sec:wp1}
\item \textbf{Until March 1st}: \nameref{sec:wp2}
\item \textbf{Until March 15th}: \nameref{sec:wp3}
\item \textbf{Until April 15th}:\nameref{sec:wp4}
\item \textbf{Until May 10th}: \nameref{sec:wp5}
\end{itemize}

  \section{Conclusion}
  \label{sec:conclusion}

The proposed research will evaluate if and how spatial narrative can be combined with semantic closeness to form visual presentations that will facilitate knowledge transfer. The knowledge presentation systen \sys that we will implement will allow us to automatically create these visual presentations from the OMDoc format. Through using semantic closeness the presentations can be created with meaningful visual connections between information. Similarly to how the Semantic Web tries to add relations between different resources for machines, \sys will add meaningful relations between course content.\\

Going back to our contrived example, \sys will provide Annie's teacher with a tool that allows her to use her normal lecture material in a more interactive way. Thus when the teacher realizes that Annie needs to revise previous material, the teacher can easily go to that material and provide Annie with a good learning experience. This facilitates knowledge transfer and allows for a more engaging and interactive education.\\

But not only will \sys make courses more engaging, interactive, and easier to follow. It will also lead to better knowledge absorption because students learn the concepts and in addition how they relate to one another. This allows faster revision of material and facilitates studying without a teacher present. \\

In essence, this proposed research will be beneficial for education and knowledge transfer. It aims to make the creation of engaging and interactive presentations simple and fast for current users of OMDoc. Should this research prove successful it could be adapted for other areas. In the future creating a meaningful, interactive presentation might be just a matter of uploading a document.

\newpage
\section{References}

\bibliography{kwarc}{}
\bibliographystyle{alpha}
\end{document}