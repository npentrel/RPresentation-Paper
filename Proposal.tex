\documentclass[twoside]{article}
\usepackage{amsmath}
\usepackage{lipsum} % Package to generate dummy text throughout this template
\usepackage[english]{isodate}% http://ctan.org/pkg/isodate
\usepackage{pgfgantt}
\usepackage[sc]{mathpazo} % Use the Palatino font
\usepackage[T1]{fontenc} % Use 8-bit encoding that has 256 glyphs
\linespread{1.5} % Line spacing - Palatino needs more space between lines
\usepackage{microtype} % Slightly tweak font spacing for aesthetics
\usepackage{minted}
\usemintedstyle{autumn}
\usepackage[hmarginratio=1:1,top=32mm,left=25mm,right=25mm,columnsep=18pt,bottom=27mm]{geometry} % Document margins
\usepackage{multicol} % Used for the two-column layout of the document
\usepackage[hang, small,labelfont=bf,up,textfont=it,up]{caption} % Custom captions under/above floats in tables or figures
\usepackage{booktabs} % Horizontal rules in tables
\usepackage{float} % Required for tables and figures in the multi-column environment - they need to be placed in specific locations with the [H] (e.g. \begin{table}[H])
\usepackage{hyperref} % For hyperlinks in the PDF

\usepackage{lettrine} % The lettrine is the first enlarged letter at the beginning of the text
\usepackage{paralist} % Used for the compactitem environment which makes bullet points with less space between them

\usepackage{abstract} % Allows abstract customization
\renewcommand{\abstractnamefont}{\normalfont\bfseries} % Set the "Abstract" text to bold
\renewcommand{\abstracttextfont}{\normalfont\small\itshape} % Set the abstract itself to small italic text

\usepackage{titlesec} % Allows customization of titles
\renewcommand\thesection{\Roman{section}} % Roman numerals for the sections
\renewcommand\thesubsection{\Roman{subsection}} % Roman numerals for subsections
\titleformat{\section}[block]{\large\scshape\centering}{\thesection.}{1em}{} % Change the look of the section titles
\titleformat{\subsection}[block]{\large}{\thesubsection.}{1em}{} % Change the look of the section titles

\usepackage{fancyhdr} % Headers and footers
\pagestyle{fancy} % All pages have headers and footers
\fancyhead{} % Blank out the default header
\fancyfoot{} % Blank out the default footer
\fancyhead[C]{Naomi Pentrel $\bullet$ \href{mailto:n.pentrel@jacobs-university.de}{n.pentrel@jacobs-university.de} $\bullet$ \today} % Custom header text
\fancyfoot[RO,LE]{\thepage} % Custom footer text
\usepackage{graphicx}
\usepackage{wrapfig}

\usepackage{tocloft}

%\usepackage{csquotes}
%\MakeOuterQuote{"}

\hypersetup{
    bookmarks=true,
    unicode=false,
    pdftoolbar=true,
    pdfmenubar=true,
    pdffitwindow=false,
    pdfstartview={FitH},
    pdftitle={My title},
    pdfauthor={Author},
    pdfsubject={Subject},
    pdfcreator={Creator},
    pdfproducer={Producer},
    pdfkeywords={keyword1} {key2} {key3},
    pdfnewwindow=true,
    colorlinks=false,
    linkcolor=red,
    citecolor=green,
    filecolor=magenta,
    urlcolor=cyan
}

\usepackage{caption}
\usepackage{subcaption}
\usepackage[section]{placeins}
\usepackage{qtree}
\usepackage{gensymb}

\usepackage{setspace}

\usepackage{color}

%\doublespacing
% or:
%\onehalfspacing

%\renewcommand\cftchapafterpnum{\vskip10pt}
%\renewcommand\cftsecafterpnum{\vskip15pt}

%----------------------------------------------------------------------------------------
%	TITLE SECTION
%----------------------------------------------------------------------------------------

\title{\vspace{-15mm}\fontsize{24pt}{10pt}\selectfont\textbf{// All comments are NOT created equal}} % Article title


%----------------------------------------------------------------------------------------

\begin{document}
\thispagestyle{empty}
\pagenumbering{roman}
\begin{flushright}
    \includegraphics[scale=0.7]{Logo}
  \end{flushright}
  \vspace{20mm}
  \begin{center}
    \huge
    \textbf{Visualization of Theory Graphs of Semantically Annotated Documents in the Form of Spatial Presentations}
  \end{center}
  \vspace*{4mm}
  \begin{center}
   \Large by
  \end{center}
  \vspace*{4mm}
  \begin{center}
    \Large
    \textbf{Naomi Pentrel}
  \end{center}
  \vspace*{20mm}
  \begin{center}
    \large
    Bachelor Thesis Proposal in Computer Science
  \end{center}
  \vfill
  \begin{flushright}
    \large
    \begin{tabular}{l}
      
      \hline
      Prof. Dr. Michael Kohlhase \\
      \\
    \end{tabular}
  \end{flushright}
  \vspace*{8mm}
  \begin{flushleft}
    \large
    Date of Submission: \today \\
    \rule{\textwidth}{1pt}
  \end{flushleft}
  \begin{center}
    \Large Jacobs University Bremen - School of Engineering and Science
  \end{center}

\newpage
\noindent
  With my signature, I certify that this thesis has been written by me
  using only the indicates resources and materials. Where I have
  presented data and results, the data and results are complete,
  genuine, and have been obtained by me unless otherwise acknowledged;
  where my results derive from computer programs, these computer
  programs have been written by me unless otherwise acknowledged. I
  further confirm that this thesis has not been submitted, either in
  part or as a whole, for any other academic degree at this or another
  institution.

  \vspace{20mm}

    \includegraphics[scale=0.2]{Signature}
 \hfill Bremen, \today
  
\newpage

\thispagestyle{fancy} % All pages have headers and footers

%----------------------------------------------------------------------------------------
%	ARTICLE CONTENTS
%----------------------------------------------------------------------------------------

 \section*{Abstract}
  
  Consider this a separate document, although it is submitted together
  with the rest. The abstract aims at another audience than the rest
  of the proposal. It is directed at the final decision maker or
  generalist, who typically is not an expert at all in your field, but
  more a manager kind of person. Thus, don't go into any technical
  description in the abstract, but use it to motivate the work and to
  highlight the importance of your project.

  (target size: 15-20 lines)

  \newpage
  \tableofcontents

  \clearpage
  \pagenumbering{arabic}

  \section{Introduction}

  This, like the rest, addresses fellow experts from your field (but
  not from your particular topic of research). Here you should
  technically connect to the main concepts from that field and give an
  outline of your project, stating the research/engineering question
  that you want to get answered by your project.



  (target size: 1 page)
\newpage
  \section{Statement and Motivation of Research}

  This part should make clear which question, exactly, you are
  pursuing, and why your project is relevant/interesting. This is the
  place to cite relevant literature. Where does your project extend
  the state of the art? what weaknesses in known approaches to you
  hope to overcome? If you have carried out preliminary experiments,
  describe them here.

What is the idea?
Why is this important for the world?

What is Prezi?
How will this improve learning?
What's the best representation one could do


  (target size: 3-5 pages)
\newpage
  \section{Planned Investigation}

  This is the technical core of the proposal. Here you lay out your
  plans of how you want to answer your research question specify your
  design of experiments or simulations, point out difficulties that
  you expect to encounter, etc.

3. Planning
How will this be implemented?
Using which frameworks?


  (target size: 3-4 pages)
\newpage
  \section{Evaluation Criteria}

  This section discusses criteria which can be used to evaluate the
  research results and which can be used to related research results
  to the already known state-of-the-art.

  (target size: 1-2 page)
\newpage
  \section{Timeline}

\textbf{Milestones}
\begin{itemize}
\item \textbf{Until February 15th}: Complete preliminary development to create a presentation from an XML
\item \textbf{Until April 1st}: Complete development of organized presentations that are generated from ordered knowledge graphs.
\item \textbf{Until May 1st}: Testing and Performance Measurement.
\item \textbf{Until May 10th}: Complete development of organized presentations that are generated from ordered knowledge graphs.
\end{itemize}


\begin{figure}[h]
\begin{center}
\begin{ganttchart}[y unit title=0.4cm,
y unit chart=0.5cm,
vgrid,hgrid, 
title label anchor/.style={below=-1.6ex},
title left shift=.05,
title right shift=-.05,
title height=1,
bar/.style={fill=gray!50},
incomplete/.style={fill=white},
progress label text={},
bar height=0.7,
group right shift=0,
group top shift=.6,
group height=.3,
group peaks={}{}{.2}]{24}
%labels
\gantttitle{Proposed Timeline}{24} \\
\gantttitle{December}{4} 
\gantttitle{January}{4} 
\gantttitle{February}{4} 
\gantttitle{March}{4} 
\gantttitle{April}{4} 
\gantttitle{May}{4} \\
%tasks
\ganttbar{Preliminary development}{2}{10} \\
\ganttbar{Graph visualization}{11}{16} \\
\ganttbar{Testing \& Performance}{17}{20} \\
\ganttbar{Final Report submission}{21}{21.5}

%relations 
\ganttlink{elem0}{elem1} 
\ganttlink{elem1}{elem2} 
\ganttlink{elem2}{elem3} 
\end{ganttchart}
\end{center}
\caption{Gantt Chart}
\end{figure}

  \section{Conclusions}

  Summarize the main aspects of the proposal.

  (target size: 1/2 page)


\end{document}